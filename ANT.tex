\documentclass[UTF8]{ctexbook}
\usepackage[margin=2cm]{geometry}
\usepackage[T1]{fontenc}
\usepackage{amsfonts}
\newcommand\Z{\mathbb Z}
\newcommand\F{\mathbb F}
\newtheorem{question}{问题}
\newtheorem{fact}{事实}

\begin{document}

\chapter{代数数论}

\section{引子}
Fermat 对不定方程的研究,可谓代数数论之滥觞.  在讨论「数域」、「素理想」、「Abel 扩张」之类的抽象概念之前,我们将沿着 Fermat 的脚步,考察一些具体的不定方程,来体会这些抽象概念对解决经典意义上的“数论”问题的帮助.

\begin{question}
  对给定的 $n$, 求 $n = x^2 + y^2$ 的整数解.
\end{question}


如果等号的右边是 $x^2 - y^2 = (x+y)(x-y)$, 这个问题是平凡的:只需将 $n$ 写成 $n = d_1d_2$ 的形式,若 $d_1$ 与 $d_2$ 同奇偶,我们就得到了一组 $(x, y)$; 且所有的解都能如此得到.

但是 $x^2 + y^2$ 并不能进一步因式分解.  {\kaishu 真的如此吗?}  我们不妨将眼界拓宽些:$x^2 + y^2 = (x+iy)(x-iy)$.

若 $m = x_1^2 + y_1^2$, $n = x_2^2 + y_2^2$, 则 $mn = (x_1x_2-y_1y_2 + i(x_1y_2+x_2y_1))(x_1x_2-y_1y_2 - i(x_1y_2+x_2y_1)) = (x_1x_2-y_1y_2)^2 + (x_1y_2+x_2y_1)^2$.  从而只要方程对 $m$, $n$ 的解,就能构造出对 $mn$ 的解.  因此只需考虑左端是素数 $p$ 的情况.  进行一些简单的试验即可发现,$p = 1 \pmod 4$ 时总是有唯一解,而 $p = 3 \pmod 4$ 时总是无解.

回头观察 $p = (x+iy)(x-iy)$.  这可以视为在 $\Z[i] := \{x+iy : x, y \in \Z$ 中的素因数分解.  于是问题转化为:什么情况下 $p$ 在 $B = \Z[i]$ 中仍是“素数”,什么情况下 $p$ 可以进一步分解?运用一些抽象代数的知识,考虑环 $\Z[x]$.  若先模去主理想 $(x^2+1)$ 再模去 $(p)$ 可得 $\bar B = \Z[i]/(p)$.  若先模去 $(p)$ 再模去 $(x^2+1)$ 则得到 $\bar B' = \F_p[x]/(x^2+1)$.  稍加考虑即可得知 $\bar B = \bar B'$, 并且有如下对应关系:

\begin{table}[h]
  \begin{tabular}{llll}
    $(p)$ 在 $\Z[i]$ 中是素理想 & $\bar B$ 是整环 & $\bar B'$ 是整环 & $(x^2+1)$ 在 $\F_p$ 上不可约 \\
    $(p)$ 在 $\Z[i]$ 中分解为相异素理想的乘积 & $\bar B$ 有零因子 & $\bar B'$ 有零因子 & $(x^2+1)$ 在 $\F_p$ 中有相异根 \\
    $(p)$ 在 $\Z[i]$ 中的分解含有某素理想的幂 & $\bar B$ 有幂零元 & $\bar B'$ 有幂零元 & $(x^2+1)$ 在 $\F_p$ 中有重根
  \end{tabular}
\end{table}

关于 $x^2+1$ 在 $\F_p$ 中解的情况,二次互反律\footnotemark{}已经为我们提供了解答——正好与我们的观察相符.  另外 $x^2+1$ 有重根当且仅当 $\Delta = -4 = 0 \pmod p$, 即 $p = 2$.  事实上在 $\Z[i]$ 中成立 $(2) = (1+i)^2$.

\footnotetext{或者注意到 $x^2+1$ 的根恰是 $\F_p^\times$ 中的 $4$ 阶元,又由 $\F_p^\times$ 是循环群可知这样的元素存在当且仅当 $p = 1 \pmod 4$.}

至此,问题已经解决了.  {\kaishu 真的如此吗?}  并不!前面的论述都只在「理想」的层面进行,而 $p = (x+iy)(x-iy)$ 的解是在「元素」层面考虑的.  因此我们还需要考虑「理想」与「元素」间的差异.  并非所有元素都能生成真理想:单位元(可逆元)生成整个环.  另外理想也未必由单个元素生成.  我们需要考虑 $\Z[i]$ 中理想的结构,并考虑其中的单位元的影响,才算是完成了对原方程解的分析.  我们有如下事实:
\begin{fact}
  $\Z[i]$ 是主理想整环.
\end{fact}
\begin{fact}
  $\Z[i]$ 中的单位元仅有 $\{\pm 1, \pm i\}$.
\end{fact}

这两个事实在一般数域中的推广以「理想类群」和「单位群」的概念来表达.  关于这两个群的重要结果——类数有限定理和 Dirichlet 单位定理——是代数数论中最重要的定理.  而关于理想分解的一般理论,则是类域论所要解决的主要问题.
\end{document}
